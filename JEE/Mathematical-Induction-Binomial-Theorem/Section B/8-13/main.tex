
\iffalse
  \title{Assignment 1}
  \author{Jadhav Rajesh}
  \section{mains}
\fi

%   \begin{enumerate}
        \item Let $S\brak{K}=1+3+5\dots+\brak{2K-1}=3+k^2$.Then which of the following 
              is true
              \hfill(2004)
    \begin{enumerate}
        \item Principal of mathematical statement can be used to prove this formula\\
        \item S\brak{K}$\Rightarrow$S\brak{K+1}\\
        \item S\brak{K}$\nRightarrow$S\brak{K+1}\\
        \item $S\brak{1}$ is correct\\
    \end{enumerate}
    \item The coefficient of the middle term in the binomial expansion in powers of $x$ of $\brak{1+\alpha x}^{4}$ and of $\brak{1-\alpha x}^6$ is the same if equals
         \hfill(2004)
    \begin{enumerate}
         \item$\frac{3}{5}$\\
         \item$\frac{10}{3}$\\
         \item$\frac{-3}{10}$\\
         \item$\frac{-5}{3}$\\
    \end{enumerate}
    \item The coefficient of $x^{n}$ in expansion of $\brak{1+x}\brak{1-x}^{n}$ is
         \hfill(2004)
    \begin{enumerate}
        \item$\brak{-1}^{n-1}n$\\
        \item$\brak{-1}^{n}\brak{1-n}$\\
        \item$\brak{-1}^{n-1}\brak{n-1}^{2}$\\
        \item$\brak{n-1}$\\
    \end{enumerate}
    \item The value of $^{50}C_{4}+\sum_{r=1}^{6}$$^{56-r}C_{3}$ is
        \hfill(2005)
    \begin{enumerate}
        \item$^{55}C_{4}$\\
        \item$^{55}C_{3}$\\
        \item$^{56}C_{3}$\\
        \item$^{56}C_{4}$\\
    \end{enumerate}
    \item If A=
                 $\myvec {
                 1 & 0\\
                 1 & 1\\
                 }$
                       and I=
                          $\myvec{
                          1 & 0\\
                          0 & 1\\
                          }$
                    ,then which of the following holds for all $n\ge1$,by the principle of mathematical induction
                    \hfill(2005)
                    \begin{enumerate}
        \item$A^{n}=nA-\brak{n-1}I$\\
        \item$A^n=2^{n-1}A-\brak{n-1}I$\\
        \item$A^n=nA+\brak{n-1}I$\\
        \item$A^n=2^{n-1}+\brak{n-1}I$\\
    \end{enumerate}
    
    \item If the coefficient of $x^{7}$ in $\sbrak{ax^{2}+\brak{\frac{1}{bx}}}^{11}$ equals the coefficient of $x^{-7}$ in $\sbrak{ax-\brak{\frac{1}{bx^2}}}^{11}$  , then $a$ and $b$ satisfy the relation
       \hfill(2005)
     \begin{enumerate}
        \item $a-b=1$\\
        \item $a+b=1$\\
        \item$\frac{a}{b}=1$\\
        \item $ab=1$\\
    \end{enumerate}

% \end{enumerate}
